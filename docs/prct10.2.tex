\documentclass[spanish,a4paper,10pt]{article}

\usepackage{latexsym,amsfonts,amssymb,amstext,amsthm,float,amsmath}
\usepackage[spanish]{babel}
\usepackage[latin1]{inputenc}
\usepackage[dvips]{epsfig}
\usepackage{doc}
\usepackage{graphicx}

%%%%%%%%%%%%%%%%%%%%%%%%%%%%%%%%%%%%%%%%%%%%%%%%%%%%%%%%%%%%%%%%%%%%%%
%123456789012345678901234567890123456789012345678901234567890123456789
%%%%%%%%%%%%%%%%%%%%%%%%%%%%%%%%%%%%%%%%%%%%%%%%%%%%%%%%%%%%%%%%%%%%%%
%\textheight 29cm
%\textwidth 15cm
%\topmargin -4cm
%\oddsidemargin 5mm

%%%%%%%%%%%%%%%%%%%%%%%%%%%%%%%%%%%%%%%%%%%%%%%%%%%%%%%%%%%%%%%%%%%%%%

\begin{document}
\title{Aproximaci�n del n�mero $\pi$ con una m�quina de c�mputo}
\author{T�cnicas Experimentales \\ Pr�ctica de Laboratorio \#5}
\date{5 de marzo de 2014}

\maketitle

\begin{abstract}
El objetivo de esta pr�ctica es entregar un programa escrito en LaTEx
en el que se hable sobre $pi$ y se aprenda a utilizar mejor dicho programa.
\end{abstract}

%\thispagestyle{empty}
%++++++++++++++++++++++++++++++++++++++++++++++++++++++++++++++++++++++
\section{Motivaci�n y Objetivos}

$pi$ es la relaci�n entre la longitud de una circunferencia y su di�metro, 
en geometr�a euclidiana. Es un n�mero irracional y una de las constantes matem�ticas m�s importantes.
Se emplea frecuentemente en matem�ticas, f�sica e ingenier�a. El valor num�rico de ?, truncado a sus 
primeras cifras, es el siguiente:

$pi$ $$ = 3,14159265358979323846 $$

$\pi$ se puede calcular mediante integraci�n:

$$\int_{0}^{1} \! \frac{4}{1+x^2}\, dx = 4(atan(1) -atan(0)) = \pi $$

Esta integral se puede aproximar num�ricamente con una f�rmula de cuadratura.
%
Si se utiliza la regla del punto medio se obtiene:

\begin{center}
$ \pi \approx \frac{1}{n} \sum\limits_{i=1}^{n}f(x_i)\,$,
con $f(x) = \frac{4}{(1+x^2)}\,$,
$x_i = \frac{i - \frac{1}{2}}{n}$,
para $i = 1, \dots, n$
\end{center}

%++++++++++++++++++++++++++++++++++++++++++++++++++++++++++++++++++++++
\subsection{La historia de pi}
hola

%+++++++++++++++++++++++++++++++++++++++++++++++++++++++++++++++++++++
\section{Ejercicios propuestos}

Escriba un programa que reciba como entrada el n�mero de subintervalos
con los que se desea abordar la aproximaci�n de $\pi$.

%++++++++++++++++++++++++++++++++++++++++++++++++++++++++++++++++++++++
\subsection{Subsecci�n 2}
hola

%+++++++++++++++++++++++++++++++++++++++++++++++++++++++++++++++++++++

\section{Tablas}


\begin{table}[!h]
\begin{center}
\begin{tabular}{|c|c|} \hline
\textbf{Tiempo } & \textbf{Velocidad} \\
\textbf{($\pm$ 0.001 s)} & \textbf{($\pm$ 0.1 m/s)} \\ \hline \hline
1.234 &
67.8
\\
\hline

2.345 &
78.9
\\
\hline

3.456 &
89.1
\\
\hline

4.567 &
91.2
\\
\hline

\end{tabular}
\end{center}
\caption{Resultados experimentales de tiempo (s) y velocidad (m/s)}
\label{tab:1}
\end{table}

\begin{table}[!h]
\begin{center}
\begin{tabular}{|l||c|c|}
\hline
Nombre & Edad & Nota \\ \hline
Pepe & 24 & 10 \\ \hline
Juan & 19 & 8 \\ \hline
Luis & 21 & 9 \\ \hline
\end{tabular}
\end{center}
\caption{Mi primer cuadro de datos}
\label{tab}
\end{table}
\footnote {nota a pie de p�gina}

\newpage
%++++++++++++++++++++++++++++++++++++++++++++++++++++++++++++++
\section{Gr�ficos}

\begin{figure}[!h]
\begin{center}
\includegraphics[width=0.75\textwidth]{imagen1.eps}
\caption{Ejemplo de figura con gr ?afico}
\label{fig}
\end{center}
\end{figure}

\footnote {nota a pie de p�gina}
\begin{thebibliography}{1}
\bibitem{python} Tutorial de Python. http://docs.python.org/2/tutorial/
\end{thebibliography}

\addcontentsline{toc}{chapter}{Bibliograf�a}
\bibliographystyle{plain}


\bibliography{bibliografia}

\nocite{*}

\end{document}